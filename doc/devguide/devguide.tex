
\documentclass{article}

% enlarge the page a bit
\addtolength{\hoffset}{-1cm}
\addtolength{\textwidth}{1.6cm}

% make new commands to enforce some style
\newcommand{\asss}{\texttt{asss}}
\newcommand{\subgame}{\texttt{subgame}}

\title{\asss{} Development Guide -- 1.2.0}
%\author{Grelminar \texttt{<grelminar@yahoo.com>}}

\begin{document}

\maketitle

\section{Introduction}

If you're reading this, you probably already know that \asss{} is a
server for the multiplayer game \emph{Subspace}, written mostly in C and
Python. This document will try to help you to understand how \asss{}
works internally and how to develop for it.

There are three types of things you might want to do with \asss{}:
modify the existing source (the stuff in the core distribution), write
new modules from scratch in C, and write new modules from scratch in
Python. You're welcome to do any of those three things, depending on
your goals, but I'd like to encourage people to try to write new modules
in Python if possible, and only use C if there's a good reason for it
(efficiency concerns, linking with other libraries, etc.). Don't let the
fact that you don't know Python discourage you; it's a very easy
language to learn. Also don't be discouraged by the current
incompleteness of the Python interface to \asss{}. It will improve as
users submit requests for things that they need added to it.


\section{Building}

If you want to build all of \asss{} from scratch, there are a few
dependencies you need to be aware of: Python, version 2.2 or greater,
Berkeley DB, version 4.0 or greater, and the mysql client libraries (any
recent version should be ok). If you're building on a unix system,
you'll need to use GNU make.

The basic procedure is to edit the definitions at the top of the
provided \verb/Makefile/ to point to the directories where your
libraries are installed. After that, running \verb/make/ should build
all of \asss{}, which consists of a binary named \verb/asss/ and a bunch
of \verb/.so/ files containing the modules. Running \verb/make install/
will copy those binaries to the \verb/bin/ directory one level up.

If you're missing one or more of those libraries, you can still build
the remaining parts of \asss{}: If you're missing Python, remove
\verb/pymod.so/ from the list of stuff to build (the variable
\verb/ALL_STUFF/). If you're missing mysql, remove \verb/database.so/.
If you're missing Berkeley DB, remove \verb/persist.so/.


\subsection{Building on FreeBSD}

FIXME

\subsection{Building on Windows}

FIXME



\section{Basic Architecture}

I had several goals when designing \asss{}: It should be modular, so
that server admins could plug in their own custom functionality in
addition to or in place of any part of the server. It should support
runtime loading, so functionality could be added, removed, and upgraded
without taking down the server. It should be robust and efficient.

Those goals led to a design that might look a little scary at first, but
is actually pretty simple if you put a little effort into understanding
it. However, there's a lot of indirection, and it can be difficult to
understand the control flow in certain places, because of the pervasive
use of callbacks. Hopefully this document can provide enough information
that anyone can understand how it all works, and more importantly, can
figure out how to modify or extend it to do what they want.

The three main pieces of the architecture are modules, interfaces, and
callbacks.

\subsection{Modules}

Almost all of the code in \asss{} is part of a module (just about
everything except \verb/main.c/, \verb/module.c/, \verb/cmod.c/, and
\verb/util.c/). A module is just a piece of code that runs as part of
the server. Modules can currently be written in either C or Python.

Some examples of modules are \verb/core/, which manages player logins
and other really important bits, \verb/flags/, which manages the flag
game, \verb/buy/ which provides an implementation of the \verb/?buy/
command, \verb/pymod/ which allows Python modules to exist, and
\verb/persist/, which provides database services for the rest of the
server.

Modules written in C have a single entry point function.

Modules by themselves can't do very much. In order to be useful, modules
have to talk to other modules. The two main ways for modules to
communicate are interfaces and callbacks.

\subsection{Interfaces}

An interface in \asss{} is just a set of function signatures. They're
implemented by C structs containing function pointers (and rarely,
pointers to other types of C data). Each interface has an identifier (a
string, although a C macro is used to hide the actual value of the
string), and the identifier contains a version number. If the contents
of an interface is changed, the version number should be incremented.

Interfaces are used for two slightly different purposes in \asss{}: they
are used for exporting functionality from one module to others, and they
are used for customizing a specific part of the server's behavior. Both
uses used the same set of functions, although in slightly different
ways, so you should be aware of the differences.

The module manager (one of the pieces of \asss{} that isn't in a module
itself) manages interface pointers for the whole server. It has several
available operations, which are exposed through an interface of its own:

\begin{itemize}

\item A module can register an interface for other modules to use. To do
this, it creates a struct and initializes its fields with pointers to
the functions it's going to use to implement the interface. (Almost
always , this struct will be statically allocated.) A special macro is
used to provide the identifier of the interface that this struct is
going to implement, and also to provide a unique name for this
implementation. Then the \verb/RegInterface/ function of the module
manager interface is called.

An interface can be registered globally for the whole server, or
registered for a single arena only.

\item A module can unregister an interface that it has previously
registered, using \verb/UnregInterface/. The same arena pointer that is
passed into \verb/RegInterface/ should be passed into this function.
Note that unregistering an interface can fail! See below about reference
counts.

\item A module can request a pointer to an implementation of an
interface, given the interface identifier, using \verb/GetInterface/.

\item A module can request a pointer to a specific implementation of an
interface, with \verb/GetInterfaceByName/.

\item A module can return a reference to an interface that it acquired
with one of the previous two functions, using \verb/ReleaseInterface/.

\end{itemize}

\subsubsection{Reference counts}

Implementations of interfaces are reference counted. A module that calls
either of the \verb/GetInterface/ calls that returns a valid pointer
owns a reference to that implementation, and must later return it with
\verb/ReleaseInterface/. Calling \verb/UnregInterface/ on an interface
pointer will fail if there are any outstanding references to that
pointer (and it will return the number of references).


\subsubsection{Arena-specific interfaces}

The functions \verb/RegInterface/, \verb/UnregInterface/, and
\verb/GetInterface/ all take an optional arena pointer. Interfaces that
serve only to export functionality will generally be registered globally
for the whole server, and there is only one possible implementation for
each of them. To register an interface globally, or to request a
globally registered interface, the macro \verb/ALLARENAS/ should be
passed as the arena pointer.

Interfaces that are used to select among different behaviors might be
registered per-arena. Passing a pointer to a valid arena to
\verb/RegInterface/ makes that interface pointer available only to
modules who call \verb/GetInterface/ with that arena. If a module calls
\verb/GetInterface/ with a valid arena pointer, but there is no
interface pointer with that id registered for that arena, it will fall
back to an interface registered globally with that id, if possible. That
allows a module to register a "default" implementation for an interface,
and let other modules override it for specific arenas.


\subsubsection{Priorities}

Another feature available when using the interface system to select
among different behaviors is priorities. Priorities should be used when
it is expected that multiple implementations of the same interface will
be registered globally at the same time. Currently, priorities are used
when selecting which authentication implementation to use.

An implementation of an interface may specify a priority (any positive
integer) using a variant of the macro used to specify the identifier and
implementation name. As long as all implementations of that interface
are registered with a priority, \verb/GetInterface/ will always return
the one with the highest priority (in the absence of priorities, the
last one registered will be returned).

Note that to use the priorities feature, \emph{all} implementations of
that interface must be registered with priorities.


\subsubsection{Example: declaring, using, and defining interfaces}

\paragraph{Declaring}
Here's a sample declaration of an interface, taken from \verb/core.h/:

\begin{verbatim}
#define I_FREQMAN "freqman-1"

typedef struct Ifreqman {
    INTERFACE_HEAD_DECL
    void (*InitialFreq)(Player *p, int *ship, int *freq);
    void (*ShipChange)(Player *p, int *ship, int *freq);
    void (*FreqChange)(Player *p, int *ship, int *freq);
} Ifreqman;
\end{verbatim}

The definition on the first line creates a macro that will be used to
refer to the interface identifier (which consists of the string
``\verb/freqman/'' followed by a version number). By convention,
interface id macros are named \verb/I_<something>/, and identifier
strings are \verb/<something>-<version>/.

Next, a C typedef is used to create a type for a struct. By convention,
struct types start with a capital \verb/I/ followed by the interface
name in lowercase. The first thing in the struct is a special macro
(\verb/INTERFACE_HEAD_DECL/) that sets up a few special fields used
internally by the interface manager. The three fields are declared as
function pointers using standard C syntax.

\paragraph{Using}
To call a function in this interface, a module might use code like this
(adapted from \verb/core.c/):

\begin{verbatim}
int freq = 0, ship = player->p_ship;
Ifreqman *fm = mm->GetInterface(I_FREQMAN, player->arena);
if (fm) {
    fm->InitialFreq(player, &ship, &freq);
    mm->ReleaseInterface(fm);
}
\end{verbatim}

This code declares a pointer to a freq manager interface, and requests
the registered implementation of the freq manager interface for the
arena that the player is in. If it finds one, it calls a function in it
and then releases the pointer.

The freq manager interface is of the kind used to select among alternate
behavior. For interfaces used for exporting functionality, typically a
module will call \verb/GetInterface/ for all the interfaces it needs
when it loads, and then keep the pointers until it unloads, at which
point it calls \verb/ReleaseInterface/ on all of them.

\paragraph{Defining}

This is a trivial implementation of the freq manager interface, used by
the recorder module to lock all players in spectator mode:

\begin{verbatim}
local void freqman(Player *p, int *ship, int *freq)
{
    *ship = SPEC;
    *freq = 8025;
}

local struct Ifreqman lockspecfm =
{
    INTERFACE_HEAD_INIT(I_FREQMAN, "fm-lock-spec")
    freqman, freqman, freqman
};
\end{verbatim}

First the functions that will implement the interface are defined. In
this case, one real function is being used to implement three functions
in the interface. Then a static struct is declared to represent the
implementation. The first thing in the struct initializer is a macro,
analogous to the macro used in the declaration.
\verb/INTERFACE_HEAD_INIT/ takes two arguments: the first is the
interface identifier, and the second is the unique name given to this
implementation. Alternately, \verb/INTERFACE_HEAD_INIT_PRI/ can be used,
which takes a third argument that is the priority.


\subsection{Callbacks}

Callbacks are somewhat simpler than interfaces, although they share many
features. A callback is a single function signature, along with an
identifier. Callback identifiers aren't versioned, but they probably
should be.

Like interfaces, callbacks are also managed by the module manager. They
can be registered globally or for a single arena. Unlike interfaces,
many callbacks registered to the same identifier can exist at once, and
all are used. The module manager functions dealing with callbacks are:

\begin{itemize}

\item To register a callback, use \verb/RegCallback/, which takes a
callback id, a function to call, and an arena to register it to. Like
interfaces, use \verb/ALLARENAS/ to indicate a globally registered
callback.

\item Use \verb/UnregCallback/ to unregister a callback. It should be
called with the same arguments as \verb/RegCallback/.

\item To query which callbacks are currently registered for an
identifier, use \verb/LookupResult/. They will be returned as a list.

\item After using the list, use \verb/FreeLookupResult/ to return the
memory used by the list.

\end{itemize}

Most of the time, you can use a provided macro to invoke all the
callbacks of a certain type, so you won't need to use
\verb/LookupResult/ and \verb/FreeLookupResult/ at all.


\subsubsection{Example: declaring, defining, and calling a callback}

\paragraph{Declaring}
Here's how the flag win callback is declared:

\begin{verbatim}
#define CB_FLAGWIN "flagwin"
typedef void (*FlagWinFunc)(Arena *arena, int freq);
\end{verbatim}

There's a macro (the naming convention is to start callback macro names
with \verb/CB_/), and a C typedef giving a name to the function
signature. All callbacks should return void.

\paragraph{Defining}
To register a function to be called for this event:

\begin{verbatim}
local void MyFlagWin(Arena *arena, int freq)
{
    /* ... contents of function ... */
}

/* somewhere in the module entry point */
mm->RegCallback(CB_FLAGWIN, MyFlagWin, ALLARENAS);
\end{verbatim}

\paragraph{Calling}
There is a special macro provided to make calling callbacks easier:
\verb/DO_CBS/. To use it, you must provide the callback id, the arena
that things are taking place in (or \verb/ALLARENAS/ if there is no
applicable arena), the C type of the callback functions, and the
arguments to pass to each registered function. It looks like:

\begin{verbatim}
    DO_CBS(CB_FLAGWIN, arena, FlagWinFunc, (arena, freq));
\end{verbatim}


\section{Important data structures}

There are several important structures that you'll need to know about to
do anything useful with \asss{}. This section will describe each of them
in detail.

\subsection{Player}

The \verb/Player/ structure is one of the most important in \asss{}.
There's one of these for each client connected to the server. These
structures are created and managed by the \verb/playerdata/ module. (The
details of when exactly in the connection process a player struct is
allocated is covered below, in the section on the player state machine.)

The first part of the player struct, which contains many important
fields, is actually in the format of the packet that gets sent to
players to inform them about other players. The benefit of using the
packet format directly to store those fields is that there's no copying
necessary when the packet needs to be sent, as the necessary information
is already in the right format.

The format of the player data packet, and then the main player struct,
will be given below, and then each field will be covered in detail.

\begin{verbatim}
struct PlayerData {
    u8 pktype;
    i8 ship;
    u8 acceptaudio;
    char name[20];
    char squad[20];
    i32 killpoints;
    i32 flagpoints;
    i16 pid;
    i16 freq;
    i16 wins;
    i16 losses;
    i16 attachedto;
    i16 flagscarried;
    u8 miscbits;
};

struct Player {
    PlayerData pkt;
#define p_ship pkt.ship
#define p_freq pkt.freq
#define p_attached pkt.attachedto
    int pid, status, type, whenloggedin;
    Arena *arena, *oldarena;
    char name[24], squad[24];
    i16 xres, yres;
    ticks_t connecttime;
    unsigned int ignoreweapons;
    struct PlayerPosition position;
    u32 macid, permid;
    char ipaddr[16];
    const char *connectas;
    struct {
        unsigned authenticated : 1;
        unsigned during_change : 1;
        unsigned want_all_lvz : 1;
        unsigned during_query : 1;
        unsigned no_ship : 1;
        unsigned no_flags_balls : 1;
        unsigned sent_ppk : 1;
        unsigned see_all_posn : 1;
        unsigned padding1 : 24;
    } flags;
    byte playerextradata[0];
};
\end{verbatim}

Details on the specific fields of the player data packet:

\begin{description}

\item[pktype] The type byte for the player data packet.

\item[ship] The ship that the player is in. 0 for Warbird, 8 for
spectator.

\item[acceptaudio] Whether the player is willing to accept .wav
messages.

\item[name] The player's name. Note: this field is \emph{not}
necessarily null-terminated.

\item[squad] The player's squad. Note: this field is \emph{not}
necessarily null-terminated.

\item[killpoints, flagpoints] Part of the player's score. Note that
\asss{} doesn't use these fields as the authoritative score, and in the
future, they might be unused entirely.

\item[pid] An identifier for the player. Pids are used extensively in
the game protocol, but not used much internally in the server.

\item[freq] The player's frequency.

\item[wins, losses] More parts of the score. See notes on killpoints and
flagpoints.

\item[attachedto] Contains the pid of the player that this player is a
turret on.

\item[flagscarried] The number of flags that the player is holding. This
field isn't guaranteed to be accurate, and is only used to help the
client figure out where the flags are when it first enters.

\item[miscbits] Currently, this field is used only for specifying
whether the player has a King-of-the-Hill crown or not.

\end{description}

Details on the specific fields of the player structure:

\begin{description}

\item[pkt] This is the player data packet described above.

\item[p\_ship] This ``virtual'' field refers to the ship field of pkt.

\item[p\_freq] This ``virtual'' field refers to the freq field of pkt.

\item[p\_attached] This ``virtual'' field refers to the attachedto field
of pkt.

\item[pid] The player id of the player. It should always agree with the
pid value in pkt.

\item[status] The current state of the player. See the description of
the player state machine below. State values are named with an initial
\verb/S_/.

\item[type] The client type of this player. Possible values are
\verb/T_UNKNOWN/, \verb/T_FAKE/ (a fake player created and managed by
the server, used for autoturrets), \verb/T_VIE/ (a Subspace 1.34 or 1.35
client), \verb/T_CONT/ (a Continuum client), or \verb/T_CHAT/ (a client
using the chat protocol).

\item[whenloggedin] This field is used by the player state machine to
make the proper transitions when a player is logging out.

\item[arena] A pointer to the arena that this player is in. It may be
null if the player isn't in an arena yet, or is between arenas.

\item[oldarena] This stores the previous value of arena when arena is
set to null. It's used to make sure scores and other persistent
information is saved properly when switching arenas or logging out.

\item[name] The player's name, guaranteed to be null terminated.

\item[squad] The player's squad, guaranteed to be null terminated.

\item[xres, yres] The player's screen resolution. Only valid when arena
is not null and for standard (\verb/T_VIE/ and \verb/T_CONT/) clients.

\item[connecttime] The time when the player first connected (in ticks).

\item[position] The last known position of the player. This contains a
few self-explanatory fields: x, y, xspeed, yspeed, and bounty. It also
contains a status field, which is a bitfield of various ship equipment.

\item[macid, permid] Various identifying values provided by standard
clients.

\item[ipaddr] A textual representation of the IP address the client is
connected from.

\item[connectas] If the player has connected to a virtual server that
specifies a default arena name, this will point to that name. Otherwise
it will be null.

\item[flags] These are a bunch of one-bit flags that are used throughout
the server:

\begin{description}

\item[authenticated] If the player has been authenticated by either a
billing server or a password file.

\item[during\_change] Set when the player has changed freqs or ships,
but before he has acknowledged it.

\item[want\_all\_lvz] If the player wants optional .lvz files.

\item[during\_query] If the player is waiting for db query results.

\item[no\_ship] If the player's lag is too high to let him be in a ship.

\item[no\_flags\_balls] If the player's lag is too high to let him have
flags or balls.

\item[sent\_ppk] If the player has sent a position packet since entering
the arena.

\item[see\_all\_posn] If the player is a bot who wants all position
packets.

\end{description}

\item[playerextradata] This variable-length array is carved up by the
player manager to store per-player data for other modules in the server.
See the section on per-player data below.

\end{description}


\subsection{Arena}

Compared to players, the arena struct is relatively simple. Arenas are
often used solely by comparing pointers for equality, although there are
several useful fields:

\begin{verbatim}
struct Arena {
    int status;
    char name[20], basename[20];
    ConfigHandle cfg;
    int specfreq;
    byte arenaextradata[0];
};
\end{verbatim}

\begin{description}

\item[status] Stores the loading/unloading state of the arena. Most
arenas will be in \verb/ARENA_RUNNING/.

\item[name] This is the arena's actual name, used for displaying to
clients, keeping track of non-shared scores, and many other things.

\item[basename] This is a name derived from the arena name, but will
trailing digits stripped off (for public arenas, whose ``name'' field
contains only digits, the ``basename'' field contains the word
``public''). This is used for keeping track of shared scores and for
locating settings for the arena.

\item[cfg] This is a handle for the arena's main configuration file.
There is only one configuration file loaded by default for each arena,
although it may include other files, and modules may load different
configuration files themselves.

\item[specfreq] This field is a concession to practicality. The
``Team:SpectatorFrequency'' setting was being queried in several places
in different modules, so rather than duplicate work, this setting is
provided here for modules to use without querying the configuration
file.

\item[arenaextradata] Like ``playerextradata,'' this variable-sized
array is managed by the arena manager to provide per-arena space for
other modules.

\end{description}


\subsection{Target}

A target is a (sometimes implicit) representation of a set of players.
Currently, targets are used as a parameter to command callbacks, to
indicate who the command should be applied to, and they are also used as
parameters to some of the functions in the \verb/game/ interface, to
warp or prize some set of players at once.

In a command function, targets can be used by accessing their fields
directly, or by using the \verb/TargetToSet/ function in the
\verb/playerdata/ interface to convert the target into a simple list of
players.

Targets can be constructed simply by declaring one on the stack and
initializing its fields. They can also be dynamically allocated,
although this isn't often necessary.

Targets come in several types, some of which use additional data
(besides the type itself) to specify the set. The data is kept in a
C union, since targets can only be of one type at once.

The most trivial target is of type \verb/T_NONE/ and means the empty set
of players. A single-player target is of type \verb/T_PLAYER/ and the
\verb/p/ field of the data union points to that player. An arena target
(\verb/T_ARENA/) indicates all players in the given arena. A freq target
(\verb/T_FREQ/) means all the players on a given team in a given arena.
Another simple target, \verb/T_ZONE/, means everyone logged into the
server. Finally, an arbitrary set of players can be specified using the
\verb/T_LIST/ type, which uses the \verb/list/ field of the data union.

This is the definition of the target struct:

\begin{verbatim}
typedef struct {
    enum {
        T_NONE,
        T_PLAYER,
        T_ARENA,
        T_FREQ,
        T_ZONE,
        T_LIST
    } type;
    union {
        Player *p;
        Arena *arena;
        struct { Arena *arena; int freq; } freq;
        LinkedList list;
    } u;
} Target;
\end{verbatim}


\section{Memory management}

Memory management in \asss{} is relatively simple. Many sorts of memory
in the server, such as the global list of players and arenas, are
managed by core modules. Others, such as the links of the linked list
library, are handled by the utility library, and a module only has to
use the linked list functions.

Sometimes, though, a module will need to allocate memory to store
private data in. There are three types of memory a module will want to
allocate: some amount of space to store data for each player, space to
store data for each arena, and arbitrary chunks of memory for whatever
use. Each type will require a different way of allocating memory.

\subsection{Per-player data}

A module can call the \verb/AllocatePlayerData/ function in
\verb/playerdata/ with a number of bytes to reserve that amount of space
for each player. The value returned is a key, which can later be used to
access that memory, given a player pointer. (Valid keys are positive
integers. If the return value is negative, the allocation failed.)
Modules that have used \verb/AllocatePlayerData/ must call
\verb/FreePlayerData/ when they don't need the space anymore (typically
during unloading).

To access the data, a macro has been provided: \verb/PPDATA(player, key)/
which will return a pointer to the start of the per-player space
specified by the key, for the given player.


\subsection{Per-arena data}

This is just like per-player data, except you will use the functions
\verb/AllocateArenaData/ and \verb/FreeArenaData/ in \verb/arenaman/,
and the macro \verb/P_ARENA_DATA(arena, key)/ to access the data for a
given arena.

\subsection{Everything else}

To allocate arbitrary chunks of memory, use the functions \verb/amalloc/
and \verb/afree/, which work just like standard \verb/malloc/ and
\verb/free/, except that \verb/amalloc/ will never return \verb/NULL/
(it will halt the server with a memory error instead), and \verb/afree/
can safely be used on null pointers.

Also try to be aware of instances where data can be allocated on the
stack, which will generally be more efficient than dynamic allocation.
If the size of the data isn't known in advance, the system's
\verb/alloca/ function can be used. If you need to pass a single-element
linked list to a function, one can be constructed on the stack by
cheating a little bit with the list abstraction, although you might want
to use the lists normally to improve the clarity of your code.


\subsection{Typical usage}

Sometimes a module needs to store a large amount of data for each
player, but it's for a specific game type that's only running in a few
arenas, and it will only apply to a small number of the players in the
zone. Allocating a large amount of data with \verb/AllocatePlayerData/
will waste space in that situation, since that reserves space for every
player, not just the ones that need it. There are two possible solutions
here. One is to just do it anyway, and waste a bit of memory. The other
is to allocate only room for a pointer in the per-player data, and have
the pointer be null for players who don't need the data, and point to
valid data (allocated with \verb/afree/ when the player enters an arena)
for players who do need it. Which solution to use depends on several
factors, such as how many bytes are being allocated, and what proportion
of players are expected to need the data.

In general, if the data is more than 20-24 bytes in length and a
significant proportion of players are expected not to need the data, you
should consider using a per-player pointer and manually allocating the
bulk of the data.


\subsection{Internals}

Per-player and per-arena data work by allocating a big chunk of space at
the end of the player and arena structs, which is divided up between
modules to store the data. As an example, let's say \verb/playerdata/
allocates 4096 bytes of extra space along with each player struct (the
exact amount is configurable). It might provide bytes 128-192 of that
space to one module to use for its private data, and then bytes 192-204
to another module. The offset of the range from the start of the extra
data array is the key returned to client modules, and the macro simply
adds that offset to the start of the array.

This solution is simple and efficient. The only disadvantage is that the
amount of extra data for players, and for arenas, is determined at
server bootup, and can't be increased or decreased while the server is
running. This may lead to situations where a module can't be loaded at
runtime because there isn't enough room in the per-player space left. It
might also waste a significant amount of space if 4k is allocated for
each player but only 1.5k is used by the loaded of modules. An admin
particularly concerned about memory might want to check the amount of
per-player and per-arena space used by some desired set of modules, then
set the allocated amounts to be slightly more than those (just to be on
the safe side).


\section{Threading}

\asss{} is a multithreaded program, and will generally have several
threads of execution doing important things at the same time. You don't
need to know all the threads and their functions to write a module, but
you do have to be aware of concurrency issues in shared data.

The most important shared data are the global lists of players and
arenas. The player list, managed by \verb/playerdata/, is protected by a
read-write lock, and you must acquire it before iterating through the
list (the same lock protects a few other items, like player status
values). The arena list is also protected with a read-write lock,
managed by \verb/arenaman/.

FIXME: write more here


\subsection{How to use threads in a module}

First, consider well whether you really need a separate thread. Possible
good reasons to use threads in your module are: it \emph{greatly}
simplifies the implementation of some aspect of the module, or the
module makes unavoidable calls that can potentially block for a very
long time.

How long a call needs to block before you should consider using threads
depends on the path on which that call is made. If it only blocks for a
long time during module load, like \verb/gethostbyname/ in the
\verb/directory/ module, then there's no need for a thread. If it's
something that happens relatively often, like writing to a file during
on every position packet, as in the \verb/record/ module, a thread is
probably called for.

If you've decided that you need threads, you can simply go ahead and use
any of the pthreads library to create and synchronize your threads. A
simple synchronized queue for message passing between threads is part of
the utility library, and may be useful for modules that deal with
threads.


\subsection{Internals}

The main thread in \asss{} runs in the main loop module, and is used for
running timer events. The \verb/net/ module has three threads of its
own, one for receiving packets from the network and processing
unreliable packets, one for processing reliable packets, and one for
sending packets. The logging module has one thread that runs log
handlers. The \verb/persist/ module uses one thread to do its database
work, and \verb/mysql/ uses a thread to communicate with the mysql
database server. The \verb/record/ module uses one thread (dynamically
created) for each arena that is either recording or playing back a saved
game.


\section{Persistent data}

The persistent data interface is one of the most confusing parts of
\asss{}, but the concept behind it is relatively simple, so it shouldn't
be hard to use it after a little thought.

The service provided by the \verb/persist/ module is persistence of
per-player, per-arena, and global data. Some examples of things that
might use it are player scores, arena statistics (e.g., kills by ship
type), a private staff message board, player inventory (in a RPG-type
game), and a shutup timeout.

Data stored by \verb/persist/ is opaque binary data. Serialization of
the actual data that the module wants to store into a byte stream is the
responsibility of the client module. Keeping that data around between
invocations of the server and sessions of the player is the
responsibility of \verb/persist/.

To use \verb/persist/ a client module must provide some information,
along with three functions that \verb/persist/ will call to manipulate
the module's data.

The first choice is whether the persistent data is to be stored
per-player or per-arena. Note that really global data (one copy for the
whole server) counts as per-arena data.

The second choice is the scope of the data. There are two choices for
scope: either there is a single copy of the data for all arenas, or
there's a separate copy for each arena. The single copy model is
specified by \verb/PERSIST_ALLARENAS/, and the one copy model by
\verb/PERSIST_GLOBAL/. Note that either option can be specified for both
per-player and per-arena data. Per-player global data means there's one
copy of the data for each player (e.g., an inventory in an RPG spanning
multiple arenas). Per-player all-arena data means there's one copy for
each (player, arena) pair (e.g., regular scores). Per-arena global data
is simply global data; there is only one copy. Per-arena all-arena data
means there's one copy for each arena (e.g., base win statistics).

Note that you don't have to actually store data for each entity. If you
want some per-arena data stored only for a few arenas, simply return an
empty piece of memory when queried for the data for an arena to which it
doesn't apply.

The third choice is the interval that the data should be stored for.
This basically indicates when the data gets reset. There are several
intervals defined in the server: ``forever,'' which as its name implies,
never gets reset; ``per-reset,'' which is supposed to be something like
a score reset (around two weeks); ``per-game,'' which is reset at the
end of each flag or ball game (or at the staff's discretion); and
``per-map-rotation,'' which is reset when the map changes.\footnote{This
currently doesn't happen automatically when the map changes.}

Historical data for intervals before the current one is saved in the
database also, and can be queried by the appropriate tools (see the User
Guide section on querying the database).

Finally, you must provide a unique key that will differentiate your data
from data stored by other modules. A key is just a 32-bit integer.

After all that information, you'll need to write three functions (no
matter whether your data is per-player or arena and what its scope is).

The \verb/GetData/ function (you can name it whatever you want, that's
just the name of the pointer in the struct you provide to
\verb/persist/) is used to query your module for data to be written to
the database. It's called when a player leaves an arena or disconnects
from the server, or when an arena is being unloaded, to save the data
from that entity before it's gone, and it's also called periodically
every few minutes, to make sure the data on disk is relatively recent,
in case of a server crash.

When \verb/GetData/ is called, the client module should serialize its
data into the buffer passed into the function, and then return the
length of the serialized data. Returning zero indicates that it has no
data to store for this entity.

The \verb/SetData/ function is called when a player logs in or enters an
arena, or when an arena is created. When called, the client module
should deserialize data from the provided buffer into whatever form it
will use it in.

\verb/ClearData/ is called before \verb/SetData/ and can be used to
clean up memory from the previous version of the data. When called, the
client module should set the relevant data to starting values, as if a
player or arena with no previously recorded data is being created.
\verb/ClearData/ will also be called when an interval ends (immediately
after a \verb/GetData/ call to get the last version of the data), to
clear all data for the new interval.


Finally, you pack up all that information and pointers to your functions
in a statically allocated and initialized struct (of type
\verb/PlayerPersistentData/ or \verb/ArenaPersistentData/), and call
\verb/persist->RegPlayerPD/ or \verb/persist->RegArenaPD/. The
\verb/persist/ module will be calling your getters and setters from its
own thread, so you should use whatever locking is necessary to ensure
correctness.


\section{The Python interface}

FIXME


\section{Misc. internals}

\subsection{The player state machine}

FIXME

\subsection{The arena state machine}

FIXME



\section{Reference}

\subsection{Source code files}

FIXME

\subsection{Interfaces}

FIXME

\subsection{Callbacks}

FIXME

\subsection{The utility library}

FIXME




\section{Tutorials}

\subsection{\texttt{log\_console}}

FIXME

\subsection{\texttt{logman}}

FIXME



\end{document}

% vim: et

